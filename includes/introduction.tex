
Odporúčacie systémy sú v dnešnej dobe čoraz viac používané v rôznych odvetviach informatiky. Každý kto využíva internet sa s nimi už pravdepodobne stretol či už vedome, alebo nevedome. Najčastejšie sa s nimi bežný človek môže stretnúť pri používaní sociálnych sietí, pozeraní videí na YouTube, pozeraní filmov na streamovacích službách, hľadaní známosti na Tinderi, či nakupovaní na Amazone. Pri všetkých spomenutých službách a ich odporúčacích systémoch ide v širokej podstate o jeden a ten istý cieľ. Zúžiť celú svoju ponuku produktov na tie, o ktoré bude mať spotrebiteľ podľa systému najpravdepodobnejšie záujem. \par
Filmový priemysel patrí medzi oblasti, kde sa vo veľkej miere využívajú odporúčacie systémy. Cieľom tejto bakalárskej práce je vytvoriť mobilnú aplikáciu s využitím odporúčacích systémov tak, aby vedela dvom používateľom na základe ich predchádzajúcich preferencií odporučiť film, ktorý bude pre oboch zaujímavý. Používatelia si potom nezávisle od seba môžu v navrhovaných filmoch pomocou jednoduchého swipeovania (z angl. slova swipe - potiahnutie prstom) vyberať, či sa im daný film páči, alebo nie, pričom ak nastane medzi nimi zhoda a obaja označia ten istý film, aplikácia im tieto zhody zobrazí. Každý používateľ aplikácie bude mať svoju vlastnú filmotéku, teda zoznam filmov ktoré ohodnotil či už pozitívne, alebo negatívne. Aplikácia bude naprogramovaná cez populárny framework React Native. \par
Motiváciou k tvorbe tejto aplikácie je problém, s ktorým sa stretlo mnoho ľudí, či už pri nekonečnom výbere večerného programu v domácnosti, alebo neúspešnom výbere spoločného filmu medzi kamarátmi. Hlavný prínos aplikácie je najmä skrátenie času pri výbere spoločného filmu, pričom aplikácia nahradí neefektívne ručné prehladávanie databáz ako ČSFD a IMDb. \par
